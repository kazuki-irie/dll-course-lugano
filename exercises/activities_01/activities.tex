\documentclass[a4paper,11pt]{article}
\usepackage[margin=.8in]{geometry}

\usepackage[utf8]{inputenc}

\usepackage{amsfonts}
\usepackage{amsmath}
\usepackage{amsthm}
\usepackage{hyperref}
\hypersetup{
    colorlinks=true,
    linkcolor=magenta
}

\title{Exercise 1}
\author{Deep Learning Lab}

\begin{document}

\maketitle

\section{Computational Resources and Basic Setups}
In this section, your task is to briefly check that you have access to all computational resources you will potentially need for this course:
\begin{enumerate}
 \item If not already done, install \href{https://www.python.org}{Python} on your \textbf{personal computer}. Make sure that you are using Python 3.
Using \href{https://packaging.python.org/tutorials/installing-packages/#use-pip-for-installing}{pip}, install the packages \texttt{numpy}, \texttt{matplotlib}, and \texttt{torch}.
Optionally, install these packages in a virtual environment, for example using
\href{https://docs.conda.io/projects/conda/en/latest/user-guide/concepts/installing-with-conda.html}{conda} by installing it from
\href{https://docs.conda.io/en/latest/miniconda.html}{here}.
\item Verify that you can access and use \href{https://colab.research.google.com}{Google Colab}.
Make sure that you know how to enable GPUs (Edit/Notebook settings/Hardware accelerator).
% \item Run the example in Slide 18 using your personal computer.
\item (Optional) Verify that you can access and use \href{https://www.kaggle.com/code}{Kaggle}. If you want to already enable access to GPUs, follow the instruction shown during the lecture.
 \item (Optional) Verify that you can use SSH to connect to USI's ICS cluster (using the credentials provided during the lecture).
The general instruction can be found \href{https://intranet.ics.usi.ch/HPC}{here}.
You will have to learn how to check the status of the cluster, submit, monitor, and kill jobs.
% \item Run the example in Slide 18 using the \textbf{ICS cluster}: adapt the script in Pg. 12 as needed, and remember to use \emph{sbatch}.
\end{enumerate}
\section{Python and NumPy}

\begin{enumerate}
 \item If you are not familiar with Python, quickly read the \href{https://docs.python.org/3/tutorial/}{Python tutorial}.
This question must be prioritized over the next question!
 \item Follow the \href{https://docs.scipy.org/doc/numpy/user/quickstart.html}{NumPy quickstart tutorial}. You should be able to:
 \begin{itemize}
  \item Create multidimensional arrays and inspect their shapes.
  \item Convert a Python list to a numpy array.
  \item Perform element-wise arithmetic operations between arrays.
  \item Perform arithmetic operations between arrays and scalars.
  \item Perform matrix multiplications using \texttt{np.dot},  \texttt{@}, or \texttt{matmul}.
  \item Perform unary operations on arrays (e.g., \texttt{max}, \texttt{sum}).
  \item Apply functions element-wise to an array (e.g., \texttt{np.sqrt}).
  \item Index elements, slice arrays, index using lists of elements, and index using Boolean arrays.
  \item Use \texttt{np.arange}.
  \item Reshape arrays. 
 \end{itemize}

% \item If you have time left, read about \href{https://docs.scipy.org/doc/numpy/user/basics.broadcasting.html}{NumPy broadcasting}. 

% \item Consider the system of linear equations represented by $\mathbf{A} \mathbf{x} = \mathbf{b}$,\\ where $\mathbf{b} = [1, 2, 3]^T$ and
% \begin{equation*}
%  \mathbf{A} = 
%    \begin{bmatrix}
%    2 & -1 & 0 \\
%    -1 & 2 & -1 \\
%    0 & -1 & 2
%    \end{bmatrix}.
% \end{equation*}
%  Find $\mathbf{x}$, using a function to invert $\mathbf{A}$ (NumPy: \emph{np.linalg.inv}, PyTorch: \emph{torch.inverse}).
\end{enumerate}
\end{document}
